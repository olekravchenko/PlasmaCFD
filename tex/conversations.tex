\NeedsTeXFormat{LaTeX2e}
\documentclass[12pt]{article}
\usepackage{graphicx}
\usepackage{natbib}
%\usepackage{upmath}
%\usepackage{amsmath}
\usepackage{amssymb}
\usepackage{amsbsy}
\usepackage{authblk}	% for affilation
\usepackage{times}
\usepackage{geometry}
 \geometry{
 a4paper,
 total={170mm,257mm},
 left=20mm,
 top=20mm,
 }

\usepackage{natbib}             % Reference
\usepackage{hyperref}           % Hyperlinks

\usepackage{datetime}		% Date
\newdateformat{specialdate}{\twodigit{\THEDAY}/\twodigit{\THEMONTH}/\twodigit{\THEYEAR}}

\usepackage{indentfirst}

\title{About simulation techniques for fluid problems}
\author{JK\&OK}
\date{\textbf{started:} 20/08/2016 \quad \textbf{current:} \specialdate\today}

\begin{document}
\maketitle

\abstract{The conversation will be towards the development of basics regarding
the development of simulations for fluid dynamical problems in any medium.}

\section{Conversations on computational fluid dynamics}
\noindent
JK: Is there any method other than dicreatization techniques for simulation? \\
OK: Finite--different, finite--volume, finite--element discretizations \cite{FDsample},\,\cite{FEMsample},\,\cite{FVMsample}. \\[1em]

\noindent
JK: So Far we discussed some aspects of  fluid simulations and that can be summarised as follows.\\

\subsection{Hydrodynamic fluid simulation}
\noindent
A. Equation of motion \cite{leveque2004}.\\
B. Equation of continuity. \cite{leveque2004}\\
C. Navier--Stokes equation. \cite{ionut2006}, \cite{pulliam2014}\\
D. Poisson equation. \cite{ionut2006}\\
Ref: \cite{ionut2006},\,\cite{leveque2004},\,\cite{lipatov2002}, \cite{pulliam2014.\\

\subsection{Kinetic simulation}
\noindent
A. Hydrodynamical with inclusion of distribution function (Yukawa, Coulomb).\\
Ref: \cite{lipatov2002}\\

\subsection{Molecular dynamics simulation}
\noindent
A. Second law of Newton with potential of interaction.\\
Ref: \cite{leimkuhler2015},\,\cite{MDsym01}\\

\noindent
JK:\\
1. How to develop a parallel 3d finite code for a fluid analytical model like Gondarenko Gujdar 2006 JGR.\\
OK: Refs?\\
2. How to develop time code run of Fig1-3 of avinash 2003 prl in colour line like fig 1 of Gondarenko-Gujdar 2006 JGR.\\  
OK: Refs? \cite[p. 280]{ionut2006}\\
\\
jk github demo dated 6 9 2016
%=====================================================================================
% Refenrece
\newpage
\bibliographystyle{plain}
\bibliography{conversations}

%=====================================================================================
% Contents
\newpage
\tableofcontents

\end{document} 